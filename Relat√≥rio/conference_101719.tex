\documentclass[conference]{IEEEtran}
\IEEEoverridecommandlockouts
% The preceding line is only needed to identify funding in the first footnote. If that is unneeded, please comment it out.
\usepackage{cite}
\usepackage{amsmath,amssymb,amsfonts}
\usepackage{algorithmic}
\usepackage{graphicx}
\usepackage{textcomp}
\usepackage{xcolor}
\def\BibTeX{{\rm B\kern-.05em{\sc i\kern-.025em b}\kern-.08em
    T\kern-.1667em\lower.7ex\hbox{E}\kern-.125emX}}
\begin{document}

\title{Pulsar\\
(Edição de halloween)}

\author{\IEEEauthorblockN{Caleb Martim}
\IEEEauthorblockA{\textit{Depto. de Ciências da Computação} \\
\textit{Universidade de Brasília}\\
Brasília, Brasil \\}
\and
\IEEEauthorblockN{Gabriel Henrique}
\IEEEauthorblockA{\textit{Depto. de Ciências da Computação} \\
\textit{Universidade de Brasília}\\
Brasília, Brasil \\}
\and
\IEEEauthorblockN{Guilherme da Rocha}
\IEEEauthorblockA{\textit{Depto. de Ciências da Computação} \\
\textit{Universidade de Brasília}\\
Brasília, Brasil \\}
}

\maketitle

\begin{abstract}
Este documento é um relatório para o projeto do grupo 6 da disciplina Introdução aos Sistemas Computacionais do semestre 2022.1 na Universidade de Brasília. O projeto é um jogo de videogame programado em Assembly na arquitetura RISC-V baseado em um outro jogo para fliperamas de 1981, Pulsar. 

Pulsar é um jogo de labirinto onde o jogador controla um tanque e seu objetivo é pegar chaves pelo mapa para desbloquear a próxima fase.$[1](https://www.arcade-museum.com/game_detail.php?game_id=9150)$

RISC-V, por vez, é um Instruction Set Architecture providenciado sob licenças de código aberto dando a liberdade para qualquer um que desejar projetar um processador baseado nele.$[2](https://riscv.org/about/)$

O jogo do projeto foi desenvolvido com suporte do RARS, um assembler para RISC-V que simula a execução de um programa em um processador RISC-V.
\end{abstract}

\begin{IEEEkeywords}
RISC-V, Assembly, RARS, Pulsar, Introdução aos Sistemas Computacionais, UnB, jogos, videogame, arcade
\end{IEEEkeywords}

\section{Introdução}
Nossa implementação do jogo tem como objetivo contar uma história fictícia sobre um fantasma que pretende fugir de uma casa assombrada por se sentir fora de lugar nela. O dever do jogador é ajudá-lo a fazer isso, coletando chaves que abrirão portas, enfrentando inimigos, até o fantasma se encontrar finalmente livre. Dessa forma, estamos implementando todas as características do jogo de 1981, Pulsar, apenas com um tema estético diferente. 

O projeto tem a intenção de ser um jogo de terror, usando músicas de suspense e sustos, incentivando o jogador a também querer ajudar o fantasma a sair da casa assombrada.

Pela apresentação do projeto acontecer no mês de outubro, nomeamos nosso jogo: Pulsar (Edição de halloween).

[Imagem para representar nosso jogo]

\section{Metodologia}

\subsection{Música e efeitos sonoros}

O jogo possui três músicas, uma de início, quando o jogador inicia o jogo; outra de derrota, quando o jogador morre no jogo; e uma de vitória, quando o jogador vence o jogo.

Cada uma dessas teve que ser manualmente composta usando as ferramentas disponibilizadas no RARS. A partir de syscalls, sons podem ser reproduzidos e dessa forma uma música pode ser simulada, com acordes, ritmo e harmonia.

As músicas em si são todas compostas de sons e músicas culturalmente famosas. A música de derrota é uma derivação do efeito sonoro de suspense famoso "Dun dun duuun!"$[3](https://en.wikipedia.org/wiki/Dun_dun_duuun!\#CITEREFSuspense1942)$ (comumente nomeado assim)
logo a música de vitória é o conjunto dos sete primeiros acordes após a introdução de Marche Funèbre$[4]https://youtu.be/hZY5DBmgC_A$  

Usamos de auxílio tutoriais de piano disponibilizados no YouTube para que pudéssemos saber as notas e suas durações de cada música.


Quanto aos efeitos sonoros foi preciso apenas usar o syscall 31 do RARS, onde no registrador a2 é guardado um valor intermediário específico entre 120 à 127. Pois estes são os valores dedicados aos efeitos sonoros que o RARS possui.

[Imagem mostrando o código de uma das músicas] 

\subsection{Arte visual}

[Imagem com a tela de início]
[Imagem com o fantasma]
[Imagem com o tiro]
[Imagem com A fonte e sua inspiração]
[Imagem com a fase 1]

\subsection{Lógica do jogo}

A base utilizada para o desenvolvimento das mecânica que foram utilizadas no jogo foi a estrutura do diagrama que criamos para facilitar a direção do desenvolvimento:

[diagrama que criamos]

Por meio do diagrama, foi possível desenvolver a estrutura básica para a troca entre as diferentes etapas do jogo, que são: Tela de Início, Fase 1, Fase 2, Tela de Morte, Tela de Vitória.

Outra mecânica importante definida a partir do diagrama foram as condições que levariam para as mudanças entre as diferentes etapas, o que guiou o desenvolvimento do trabalho.

Para o desenvolvimento das mecânicas internas do jogo, as quais não foram claramente definidas no diagrama base, foram desenvolvidos diferentes métodos para atingir os objetivos desejados. As mecânicas, com seus respectivos métodos, foram exemplificadas nos próximos tópicos.

\subsubsection{Sistema de colisão}
 O sistema de colisões e coletas de chaves foram baseados na análise das cores dos pixeis nos arredores dos personagnes, tiros e inimigos. Comparando as diferentes cores possíveis tornou-se possível determinar as reações esperadas para cada um dos casos e tornava um único código capaz de atender todos os objetivos citados

[Imagem com a representação do sistema de colisão]
\subsubsection{Movimentação do personagem} 
  A movimentação do personagem, dos inimigos e dos tiros e a limpeza de rastros foram baseados em labels que continham  os valores da posição anterior e atual dos alvos da análise. As labels continham os valores X e Y do alvo que podem ser acessados por outras partes do código, possibilitando a movimentação, limpeza de rastros e o funcionamento do sistema de colisões.
  
\subsubsection{Inteligência dos inimigos} 
  A inteligência dos inimigos criados para o jogo
\begin{itemize}
  \item O rato
  
  [Imagem com o rato se movimentando pela parede]
  
  \item O morcego
  
  [Imagem com o morcego perseguindo o personagem]
\end{itemize}
 
\subsubsection{Sistema de chaves}  
O sistema de chaves e portas foi feito com o uso de...

[Imagem com a chave e sua respectiva porta]

O Sistema de score foi feito... 

[Imagem com o Score]

\section{Resultados Obtidos}
No geral, nossas dificuldades foram superadas e conseguimos criar um jogo completo, com início, meio e fim. 

O tema que tinhamos em mente foi implementado com sucesso assim como as técnicas que elaboramos para a execução do projeto. 

Como resultado, a equipe adquiriu um melhor entendimento das mecânicas para o desenvolvimento em Assembly e a lógica de programação em Assembly na arquitetura RISC-V.

Apesar de não termos conseguido implementar uma música de fundo para tocar durante o jogo, as músicas tocadas. Além disso, por um lado pode ser visto positivamente 

Desenvolvimento de habilidades relacionadas ao uso limitado de recursos para desenvolvimento de sistemas complexos(Música, imagens, mecênicas do jogo).

\section*{Conclusão}

The preferred spelling of the word ``acknowledgment'' in America is without 
an ``e'' after the ``g''. Avoid the stilted expression ``one of us (R. B. 
G.) thanks $\ldots$''. Instead, try ``R. B. G. thanks$\ldots$''. Put sponsor 
acknowledgments in the unnumbered footnote on the first page.

\section*{Referências Bibliográficas}

Please number citations consecutively within brackets \cite{b1}. The 
sentence punctuation follows the bracket \cite{b2}. Refer simply to the reference 
number, as in \cite{b3}---do not use ``Ref. \cite{b3}'' or ``reference \cite{b3}'' except at 
the beginning of a sentence: ``Reference \cite{b3} was the first $\ldots$''

Number footnotes separately in superscripts. Place the actual footnote at 
the bottom of the column in which it was cited. Do not put footnotes in the 
abstract or reference list. Use letters for table footnotes.

Unless there are six authors or more give all authors' names; do not use 
``et al.''. Papers that have not been published, even if they have been 
submitted for publication, should be cited as ``unpublished'' \cite{b4}. Papers 
that have been accepted for publication should be cited as ``in press'' \cite{b5}. 
Capitalize only the first word in a paper title, except for proper nouns and 
element symbols.

For papers published in translation journals, please give the English 
citation first, followed by the original foreign-language citation \cite{b6}.

\begin{thebibliography}{00}
\bibitem{b1} G. Eason, B. Noble, and I. N. Sneddon, ``On certain integrals of Lipschitz-Hankel type involving products of Bessel functions,'' Phil. Trans. Roy. Soc. London, vol. A247, pp. 529--551, April 1955.
\bibitem{b2} J. Clerk Maxwell, A Treatise on Electricity and Magnetism, 3rd ed., vol. 2. Oxford: Clarendon, 1892, pp.68--73.
\bibitem{b3} I. S. Jacobs and C. P. Bean, ``Fine particles, thin films and exchange anisotropy,'' in Magnetism, vol. III, G. T. Rado and H. Suhl, Eds. New York: Academic, 1963, pp. 271--350.
\bibitem{b4} K. Elissa, ``Title of paper if known,'' unpublished.
\bibitem{b5} R. Nicole, ``Title of paper with only first word capitalized,'' J. Name Stand. Abbrev., in press.
\bibitem{b6} Y. Yorozu, M. Hirano, K. Oka, and Y. Tagawa, ``Electron spectroscopy studies on magneto-optical media and plastic substrate interface,'' IEEE Transl. J. Magn. Japan, vol. 2, pp. 740--741, August 1987 [Digests 9th Annual Conf. Magnetics Japan, p. 301, 1982].
\bibitem{b7} M. Young, The Technical Writer's Handbook. Mill Valley, CA: University Science, 1989.
\end{thebibliography}
\vspace{12pt}

\end{document}
